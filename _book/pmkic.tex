% Options for packages loaded elsewhere
\PassOptionsToPackage{unicode}{hyperref}
\PassOptionsToPackage{hyphens}{url}
%
\documentclass[
]{book}
\usepackage{lmodern}
\usepackage{amssymb,amsmath}
\usepackage{ifxetex,ifluatex}
\ifnum 0\ifxetex 1\fi\ifluatex 1\fi=0 % if pdftex
  \usepackage[T1]{fontenc}
  \usepackage[utf8]{inputenc}
  \usepackage{textcomp} % provide euro and other symbols
\else % if luatex or xetex
  \usepackage{unicode-math}
  \defaultfontfeatures{Scale=MatchLowercase}
  \defaultfontfeatures[\rmfamily]{Ligatures=TeX,Scale=1}
\fi
% Use upquote if available, for straight quotes in verbatim environments
\IfFileExists{upquote.sty}{\usepackage{upquote}}{}
\IfFileExists{microtype.sty}{% use microtype if available
  \usepackage[]{microtype}
  \UseMicrotypeSet[protrusion]{basicmath} % disable protrusion for tt fonts
}{}
\makeatletter
\@ifundefined{KOMAClassName}{% if non-KOMA class
  \IfFileExists{parskip.sty}{%
    \usepackage{parskip}
  }{% else
    \setlength{\parindent}{0pt}
    \setlength{\parskip}{6pt plus 2pt minus 1pt}}
}{% if KOMA class
  \KOMAoptions{parskip=half}}
\makeatother
\usepackage{xcolor}
\IfFileExists{xurl.sty}{\usepackage{xurl}}{} % add URL line breaks if available
\IfFileExists{bookmark.sty}{\usepackage{bookmark}}{\usepackage{hyperref}}
\hypersetup{
  pdftitle={Plan de Marketing ::\textbar:: Investigación Comercial},
  pdfauthor={Roberto Gil},
  hidelinks,
  pdfcreator={LaTeX via pandoc}}
\urlstyle{same} % disable monospaced font for URLs
\usepackage{longtable,booktabs}
% Correct order of tables after \paragraph or \subparagraph
\usepackage{etoolbox}
\makeatletter
\patchcmd\longtable{\par}{\if@noskipsec\mbox{}\fi\par}{}{}
\makeatother
% Allow footnotes in longtable head/foot
\IfFileExists{footnotehyper.sty}{\usepackage{footnotehyper}}{\usepackage{footnote}}
\makesavenoteenv{longtable}
\usepackage{graphicx,grffile}
\makeatletter
\def\maxwidth{\ifdim\Gin@nat@width>\linewidth\linewidth\else\Gin@nat@width\fi}
\def\maxheight{\ifdim\Gin@nat@height>\textheight\textheight\else\Gin@nat@height\fi}
\makeatother
% Scale images if necessary, so that they will not overflow the page
% margins by default, and it is still possible to overwrite the defaults
% using explicit options in \includegraphics[width, height, ...]{}
\setkeys{Gin}{width=\maxwidth,height=\maxheight,keepaspectratio}
% Set default figure placement to htbp
\makeatletter
\def\fps@figure{htbp}
\makeatother
\setlength{\emergencystretch}{3em} % prevent overfull lines
\providecommand{\tightlist}{%
  \setlength{\itemsep}{0pt}\setlength{\parskip}{0pt}}
\setcounter{secnumdepth}{5}
\usepackage{booktabs}
\usepackage[]{natbib}
\bibliographystyle{apalike}

\title{Plan de Marketing ::\textbar:: Investigación Comercial}
\author{Roberto Gil}
\date{2020-12-21}

\begin{document}
\maketitle

{
\setcounter{tocdepth}{1}
\tableofcontents
}
\hypertarget{presentaciuxf3n}{%
\chapter{Presentación}\label{presentaciuxf3n}}

El marketing nos rodea. Es una disciplina de reciente incorporación al plan de estudios como tal, pero que en el mundo empresarial lleva muchos años ya de camino y aplicación. Dentro de los estudios de grado relacionados con la administración de la empresa, el marketing es una disciplina que todos los estudiantes ven con muy buenos ojos por su caracterización de modernidad, dinamismo y actualidad.

Mucha bibliografía hay respecto marketing, pero sobre todos destaca un nombre, Phillip Kotler. Este autor estadounidense, economista y especialista en marketing escribió una de las definiciones más famosas de marketing, \emph{``el marketing es un proceso social y administrativo mediante el cual grupos e individuos obtienen lo que necesitan y desean a través de generar, ofrecer e intercambiar productos de valor con sus iguales''} (Philip Kotler). {[}Dirección de Mercadotecnia, Octava Edición, de Philip Kotler, Pág. 7.{]}

Otras definiciones son:

Según Jerome McCarthy, ``el marketing es la realización de aquellas actividades que tienen por objeto cumplir las metas de una organización, al anticiparse a los requerimientos del consumidor o cliente y al encauzar un flujo de mercancías aptas a las necesidades y los servicios que el productor presta al consumidor o cliente''.

Stanton, Etzel y Walker, proponen la siguiente definición de marketing: ``El marketing es un sistema total de actividades de negocios ideado para planear productos satisfactores de necesidades, asignarles precios, promover y distribuirlos a los mercados meta, a fin de lograr los objetivos de la organización'' {[}2{]}.

Para John A. Howard, de la Universidad de Columbia, ``el marketing es el proceso de:
1) Identificar las necesidades del consumidor, 2) conceptualizar tales necesidades en función de la capacidad de la empresa para producir, 3) comunicar dicha conceptualización a quienes tienen la capacidad de toma de decisiones en la empresa. 4) conceptualizar la producción obtenida en función de las necesidades previamente identificadas del consumidor y 5) comunicar dicha conceptualización al consumidor'' {[}3{]}.

Según Al Ries y Jack Trout, ``el término marketing significa''guerra``. Ambos consultores, consideran que una empresa debe orientarse al competidor; es decir, dedicar mucho más tiempo al analisis de cada''participante" en el mercado, exponiendo una lista de debilidades y fuerzas competitivas, así como un plan de acción para explotarlas y defenderse de ellas {[}3{]}.

Para la American Marketing Asociation (A.M.A.), ``el marketing es una función de la organización y un conjunto de procesos para crear, comunicar y entregar valor a los clientes, y para manejar las relaciones con estos últimos, de manera que beneficien a toda la organización\ldots{}'' {[}5{]}.

Buscar: {[}1{]}: Del libro: Dirección de Mercadotecnia, Octava Edición, de Philip Kotler, Pág. 7.
{[}2{]}: Del libro: Fundamentos de marketing, 13a Edición, de Stanton, Etzel y Walker, Pág. 7.
{[}3{]}: Del libro: La guerra de la mercadotecnia de Al Ries y Jack Trout, Págs. 4 y 5.
{[}4{]}: Del libro: El marketing según Kotler, de Philip Kotler, Edición 1999, Pág. 58.
{[}5{]}: Del sitio web de la American Marketing Asociation: MarketingPower.com, sección Dictionary of Marketing Terms, URL del sitio: \url{http://www.marketingpower.com/}
{[}6{]}: Curso Práctico de Técnicas Comerciales, ediciones Nueva Lente S.A., 2do Fascículo, pág. 25.
{[}7{]}: Del libro: Fundamentos del marketing, sexta edición, de Philip Kotler y Gary Armstrong, Pág. 21.

Tras esta definición muchas otras han llegado. Bibliografía específica de marketing inunda todas las estanterías de bibliotecas oficiales y particulares. Basta una búsqueda en Google Books del término Marketing, y las cifras asustan. 114.000 entradas bibliográficas.

Nosotros, sin embargo en esta asignatura vamos a centrar nuestro esfuerzos en una parte del Marketing, quizá la más trascendente y la que más lo caracteriza desde el punto de vista estratégico y operativo como es el Plan de Marketing. La misma búsqueda en Google Books para el término ``Plan de Marketing'' arroja el saldo de 9.900 entradas. No son unos indicadores de máxima fiabilidad, pero si muestran claramente la importancia de ambos.

\hypertarget{tema01}{%
\chapter{Tema 01 - Introducción}\label{tema01}}

El plan de marketing es un valioso instrumento que sirve de guía a todas las personas que están vinculadas con las actividades de mercadotecnia de una empresa u organización porque describe aspectos tan importantes como los objetivos de mercadotecnia que se pretenden lograr, el cómo se los va a alcanzar, los recursos que se van a emplear, el cronograma de las actividades de mercadotecnia que se van a implementar y los métodos de control y monitoreo que se van a utilizar para realizar los ajustes que sean necesarios.

Por todo ello, resulta muy conveniente que todas las personas relacionadas con el área de marketing conozcan en qué consiste el plan de marketing y cuál es su cobertura, alcance, propósitos y contenido, para que de esa manera, estén mejor capacitados para comprender la utilidad y el valor de este importante instrumento de la mercadotecnia.

Según la American Marketing Asociation (A.M.A.), el plan de marketing es un documento compuesto por un análisis de la situación de mercadotecnia actual, el análisis de las oportunidades y amenazas, los objetivos de mercadotecnia, la estrategia de mercadotecnia, los programas de acción y los ingresos proyectados (el estado proyectado de pérdidas y utilidades). Este plan puede ser la única declaración de la dirección estratégica de un negocio, pero es más probable que se aplique solamente a una marca de fábrica o a un producto específico. En última situación, el plan de marketing es un mecanismo de la puesta en práctica que se integra dentro de un plan de negocio estratégico total {[}1{]}.

Según McCarthy y Perrault, el plan de marketing es la formulación escrita de una estrategia de marketing y de los detalles relativos al tiempo necesario para ponerla en práctica. Deberá contener una descripción pormenorizada de lo siguiente: 1) qué combinación de marketing se ofrecerá, a quién (es decir, el mercado meta) y durante cuánto tiempo; 2) que recursos de la compañía (que se reflejan en forma de costes) serán necesarios, y con que periodicidad (mes por mes, tal vez); y 3) cuáles son los resultados que se esperan (ventas y ganancias mensuales o semestrales, por ejemplo). El plan de marketing deberá incluir además algunas medidas de control, de modo que el que lo realice sepa si algo marcha mal {[}2{]}.

En síntesis, el plan de marketing es un instrumento de comunicación plasmado en un documento escrito que describe con claridad lo siguiente: 1) la situación de mercadotecnia actual, 2) los resultados que se esperan conseguir en un determinado periodo de tiempo, 3) el cómo se los va a lograr mediante la estrategia y los programas de mercadotecnia, 4) los recursos de la compañía que se van a emplear y 5) las medidas de monitoreo y control que se van a utilizar.

\hypertarget{cobertura-del-plan-de-marketing}{%
\section{Cobertura del Plan de Marketing:}\label{cobertura-del-plan-de-marketing}}

El plan de marketing es un instrumento que puede servir a toda la empresa u organización, sin embargo, es más frecuente que sea elabore uno para cada división o unidad de negocios. Por otra parte, también existen ---situaciones--- en las que son imprescindibles ---planes más específicos---, por ejemplo, cuando existen marcas clave, mercados meta muy importantes o temporadas especiales (como ocurre con la ropa de moda o de temporada).

\hypertarget{alcance-del-plan-de-marketing}{%
\section{Alcance del Plan de Marketing:}\label{alcance-del-plan-de-marketing}}

Por lo general, el plan de marketing tiene un alcance anual. Sin embargo, puede haber excepciones, por ejemplo, cuando existen productos de temporada (que pueden necesitar planes específicos para 3 o 6 meses) o cuando se presentan situaciones especiales (como el ingreso de nuevos competidores o cuando se producen caídas en las ventas como consecuencia de problemas sociales o macroeconómicos) que requieren de un nuevo plan que esté mejor adaptado a la situación que se está presentando.

\hypertarget{propuxf3sitos-del-plan-de-marketing}{%
\section{Propósitos del Plan de Marketing}\label{propuxf3sitos-del-plan-de-marketing}}

El plan de marketing cumple al menos tres propósitos muy importantes:

\begin{itemize}
\tightlist
\item
  Es una ---guía escrita--- que señala las estrategias y tácticas de mercadotecnia que deben implementarse para alcanzar objetivos concretos en periodos de tiempo definidos.
\item
  Esboza ---quién--- es el responsable de ---qué--- actividades, ---cuándo--- hay que realizarlas y ---cuánto--- tiempo y dinero se les puede dedicar {[}3{]}.
\item
  Sirve como un ---mecanismo de control---. Es decir, establece estándares de desempeño contra los cuales se puede evaluar el progreso de cada división o producto {[}4{]}.
\end{itemize}

\hypertarget{el-contenido-del-plan-de-marketing}{%
\section{El Contenido del Plan de Marketing}\label{el-contenido-del-plan-de-marketing}}

No existe un formato o fórmula única de la cual exista acuerdo universal para elaborar un plan de marketing. Esto se debe a que, en la práctica, cada empresa u organización, desarrollará el método, el esquema o la forma que mejor parezca ajustarse a sus necesidades {[}4{]}.

Sin embargo, también es cierto que resulta muy apropiado el tener una idea acerca del contenido básico que debe tener un plan de marketing. Por ello, diversos autores presentan sus opciones e ideas al respecto; las cuales, se sintetizan en los siguientes puntos:

Resumen Ejecutivo: En esta sección se presenta un panorama general de la propuesta del plan para una revisión administrativa {[}5{]}. Es una sección de una o dos páginas donde se describe y explica el curso del plan. Está destinado a los ejecutivos que quieren las generalidades del plan, pero no necesitan enterarse de los detalles {[}3{]}.
Análisis de la Situación de Marketing: En esta sección del plan se incluye la información más relevante sobre los siguientes puntos {[}5{]}:
Situación del Mercado: Aquí se presentan e ilustran datos sobre su tamaño y crecimiento (en unidades y/o valores). También se incluye información sobre las necesidades del cliente, percepciones y conducta de compra {[}5{]}.
Situación del Producto: En esta parte, se muestran las ventas, precios, márgenes de contribución y utilidades netas, correspondientes a años anteriores {[}5{]}.
Situación Competitiva: Aquí se identifica a los principales competidores y se los describe en términos de tamaño, metas, participación en el mercado, calidad de sus productos y estrategias de mercadotecnia {[}5{]}.
Situación de la Distribución: En esta parte se presenta información sobre el tamaño y la importancia de cada canal de distribución {[}5{]}.
Situación del Macroambiente: Aquí se describe las tendencias generales del macroambiente (demográficas, económicas, tecnológicas, político legales y socioculturales), relacionadas con el futuro de la línea de productos o el producto {[}5{]}.
Análisis FODA-A: En esta sección se presenta un completo análisis en el que se identifica 1) las principales Oportunidades y Amenazas que enfrenta el negocio y 2) las principales Fortalezas y Debilidades que tiene la empresa y los productos y/o servicios. Luego, se define las principales Alternativas a las que debe dirigirse el plan.
Objetivos: En este punto se establecen objetivos en dos rubros {[}5{]}:
Objetivos Financieros: Por ejemplo, obtener una determinada tasa anual de rendimiento sobre la inversión, producir una determinada utilidad neta, producir un determinado flujo de caja, etc\ldots{}
Objetivos de Marketing: Este es el punto donde se convierten los objetivos financieros en objetivos de mercadotecnia. Por ejemplo, si la empresa desea obtener al menos un 10\% de utilidad neta sobre ventas, entonces se debe establecer como objetivo una cantidad tanto en unidades como en valores que permitan obtener ese margen de utilidad. Por otra parte, si se espera una participación en el mercado del 5\% en unidades, se deben cuadrar los objetivos en unidades para que permitan llegar a ese porcentaje.
Otros objetivos de marketing son: Obtener un determinado volumen de ventas en unidades y valores, lograr un determinado porcentaje de crecimiento con relación al año anterior, llegar a un determinado precio de venta promedio que sea aceptado por el mercado meta, lograr o incrementar la conciencia del consumidor respecto a la marca, ampliar en un determinado porcentaje los centros de distribución.
Cabe señalar que los objetivos anuales que se establecen en el plan de marketing deben contribuir a que se consigan las metas de la organización y las metas estratégicas de mercadotecnia {[}3{]}.
Estrategias de Marketing: En esta sección se hace un bosquejo amplio de la estrategia de mercadotecnia o ``plan de juego'' {[}5{]}. Para ello, se puede especificar los siguientes puntos:
El mercado meta que se va a satisfacer.
El posicionamiento que se va a utilizar.
El producto o línea de productos con el que se va a satisfacer las necesidades y/o deseos del mercado meta.
Los servicios que se van a proporcionar a los clientes para lograr un mayor nivel de satisfacción.
El precio que se va a cobrar por el producto y las implicancias psicológicas que puedan tener en el mercado meta (por ejemplo, un producto de alto precio puede estimular al segmento socioeconómico medio-alto y alto a que lo compre por el sentido de exclusividad).
Los canales de distribución que se van a emplear para que el producto llegue al mercado meta.
La mezcla de promoción que se va a utilizar para comunicar al mercado meta la existencia del producto (por ejemplo, la publicidad, la venta personal, la promoción de ventas, las relaciones públicas, el marketing directo).
Tácticas de Marketing: También llamadas programas de acción {[}5{]}, actividades específicas o planes de acción, son concebidas para ejecutar las principales estrategias de la sección anterior {[}3{]}. En esta sección se responde a las siguientes preguntas {[}5{]}:
¿Qué se hará?
¿Cuándo se hará?
¿Quién lo hará?
¿Cuánto costará?
Programas Financieros: En esta sección, que se conoce también como ``proyecto de estado de pérdidas y utilidades'' {[}5{]}, se anotan dos clases de información:
1) El rubro de ingresos que muestra los pronósticos de volumen de ventas por unidades y el precio promedio de venta {[}5{]}.
2) El rubro correspondiente a gastos que muestra los costos de producción, distribución física y de mercadotecnia, desglosados por categorías.
La ``diferencia'' (ingresos - egresos) es la utilidad proyectada {[}5{]}.
Cronograma: En esta sección, que se conoce también como calendario {[}3{]}, se incluye muchas veces un diagrama para responder a la pregunta ---cuándo se realizarán las diversas actividades de marketing planificadas--- {[}3{]}. Para ello, se puede incluir una tabla por semanas o meses en el que se indica claramente cuando debe realizarse cada actividad.
Monitoreo y Control: En esta sección, que se conoce también como procedimientos de evaluación, se responde a las preguntas: qué, quién, cómo y cuándo, con relación a la medición del desempeño a la luz de las metas, objetivos y actividades planificadas en el plan de marketing.
Esta última sección describe los controles para dar seguimiento a los avances {[}5{]}.
\#\# Concepto del plan de marketing

Debemos comenzar por definir qué es un plan de marketing y dentro del mismo, porqué vamos a dirigir todos nuestros esfuerzos en esta asignatura a trabajar la parte más operativa del mismo, la investigación comercial. Sainz de Vicuña lo define como ``un documento escrito con un contenido estructurado y sistematizado que define claramente los objetivos marcados por la organización y el modo de alcanzarlos fijando las responsabilidades para el desarrollo de la operativa empresarial y de los controles necesarios que permitan alcanzar la meta establecida''. Por su parte, Sanz de la Tajada indica que el plan de marketing ``es un documento escrito en el que, de una forma sistemática y estructurada, y previos los correspondientes análisis y estudios, se definen los objetivos a conseguir en un período de tiempo determinado, así como se detallan los programas y medios de acción que son precisos para alcanzar los objetivos enunciados en el plazo previsto''.

Quiero destacar en la definición de Sanz de la Tajada, su referencia a los correspondientes análisis y estudios porque esta va a ser la piedra angular de nuestro cuatrimestre. Debemos aprender a definir nuestro sistema de información en la empresa, de forma que nos aporte un umbral de certidumbre en todas aquellas decisiones de tipo estratégicos como operativo que tengamos que lanzar, tanto si estamos ante un negocio próspero, como ante una empresa que está naciendo y/o por nacer. El plan de marketing precisa de la realización de estudios de mercado que nos permitan conocer el pasado y el presente y anticipar tendencias de futuro.

También en la definición de Sainz de Vicuña, debemos resaltar algunos elementos que le dan cuerpo al plan de marketing. Se habla de él como un documento escrito y formal especificando por tanto la necesidad de dejar por escrito todas aquellas ideas que las gentes de marketing de la empresa, a quienes se les ha asignado la tarea de orientar la misma al mercado. Dejar escrito es dejar constancia de todo aquello que se ha decidido y es una guía para futuros análisis y toma de decisiones. El plan de marketing debe ser formal, comparable con otros planes de marketing en estructura, de forma que podamos hacer un análisis transversal en el tiempo comparativo entre planes de marketing en nuestra u otras empresas y longitudinal comparando diferentes periodicidades del mismo.

Del mismo modo, habla de documento estructurado, con un guión fijo que refleje tanto el entorno de la empresa de forma externa como interna, así como un diagnóstico claro de las debilidades, amenazas, fortalezas y oportunidades de la empresa. El plan de marketing deberá ser también sistematizado, formando un conjunto de unidades relacionadas entre sí mismas que ordenadamente contribuyen a un mismo objetivo finalista.

Por último, el plan de marketing determina quién y cómo realizará las acciones de tipo estratégico y/o táctica determinadas en el mismo para tratar de alcanzar los objetivos marcados en el documento y cuándo deben ser realizadas.

En definitiva, el plan de marketing:

\begin{itemize}
\tightlist
\item
  Fija objetivos
\item
  Mide la caracterización de los mismos
\item
  Desarrolla estrategias
\item
  Elige las mejores estrategias
\item
  Detalla herramientas o medios de acción
\item
  Implementa las acciones.
\item
  Traduce los objetivos a planes de acción
\end{itemize}

Si tratamos el desarrollo del plan de marketing como una acción continuada fruto del esfuerzo de todo el año, tres grandes ámbitos podemos destacar:

\begin{itemize}
\tightlist
\item
  Conocimiento de la situación de partida de la empresa, realizar una análisis de la situación y deriva del mismo un diagnóstico (DAFO) de nuestra actividad;
\item
  Marcar el camino o senda donde debemos dirigir la actividad estratégica de la empresa;\\
\item
  Definir los planes estratégicos (largo plazo) o tácticos (corto plazo) que la empresa debe ejecutar para alcanzar los objetivos marcados. Del mismo modo, el plan debe servir para establecer todos los mecanismos de control en la implementación, ejecución y posterior evaluación y revisión de los logros obtenidos.
\end{itemize}

\hypertarget{guiuxf3n-formal-de-un-plan-de-marketing}{%
\section{Guión formal de un Plan de Marketing}\label{guiuxf3n-formal-de-un-plan-de-marketing}}

Presentamos en este epígrafe el guión de entradas principales de lo que podría ser el plan de marketing de una empresa. Este guión o índice es genérico y se debería modificar para adaptarlo a la cultura y organización de productos o servicios de una empresa.
1) Resumen ejecutivo
2) Análisis de situación
a) Análisis interno
i) Presentación del producto
ii) Análisis del producto
iii) Análisis de la distribución
iv) Análisis del precio
v) Análisis de la comunicación
b) Análisis externo
i) Descripción del mercado
ii) Análisis de la competencia
iii) Tipo de consumidor objetivo
iv) Tendencias del mercado
3) Diagnóstico de situación
a) Debilidades
b) Amenazas
c) Fortalezas
d) Oportunidades
e) Vista de matriz
4) Objetivos del plan de marketing
a) Objetivo general y objetivos específicos
b) Objetivos económicos
5) Diseño del sistema de información y estudios (investigación)
a) Objetivos de la investigación en la empresa
b) Metodología/s empleada/s
c) Fases de la investigación
d) Resultados obtenidos
e) Conclusiones a los resultados
f) Recomendaciones
6) Estrategias y acciones del marketing-mix
a) Grandes líneas estratégicas.
i) Valor diferencial.
ii) La idea en la mente del consumidor.
iii) Los beneficios para el cliente.
iv) Branding.
v) Estrategia de clientes / Estrategia de lanzamiento / Estrategia de desarrollo.
b) Estrategia y acciones de producto
c) Estrategia y acciones de distribución
d) Estrategia y acciones de precio
e) Estrategia y acciones de promoción
7) Planificación temporal de acciones
8) Presupuesto
a) Estimación de ingresos
b) Estimación de gastos
9) Control
a) Control de organización
b) Control de ejecución o implementación
c) Control de rentabilidad
d) Revisiones anteriores
10) Cuadros de Previsiones
a) Ventas a corto
b) Resultados acorto
c) Ventas la largo (5 años)
d) Resultados a largo (5 años)

\hypertarget{concepto-de-investigaciuxf3n-de-mercados-comercial}{%
\section{Concepto de investigación de mercados / comercial}\label{concepto-de-investigaciuxf3n-de-mercados-comercial}}

La American Marketing Association (1987) define la Investigación Comercial como\ldots{}

\begin{quote}
\emph{``El proceso que enlaza a consumidor, cliente y público objetivo con los managers de marketing a través de la obtención de información utilizada para identificar y definir las oportunidades y problemas de marketing; generar, redefinir y evaluar las acciones de marketing; controlar su desarrollo, y fomentar el conocimiento del marketing como un proceso. La investigación comercial especifica la información requerida para orientar estas cuestiones, diseña el método de recogida de la información, dirige e implementa el proceso de recogida de datos, analiza los resultados y comunica los hallazgos y sus implicaciones''}.
\end{quote}

Esta definición, la más utilizada en los manuales de Investigación destaca todo aquello que la investigación comercial o de mercados (sinónimas) puede hacer por nosotros en el ámbito de la empresa. Podemos observar como la primera cuestión a la que hace referencia es a la del enlace de la empresa (las gentes de marketing en la empresa) con el consumidor o cliente, es decir con aquella persona que tiene el datos que nos será relevante, la opinión de referencia, el comportamiento buscado\ldots{} Esos datos nos pueden ser transferidos bajo unas normas muy estrictas que nos imponemos los que nos dedicamos a la investigación, permitiendo que nos dotemos de una umbral de certidumbre en el momento de la toma de decisiones. Esas decisiones estarán acompañadas por la información. Nótese que se utiliza un término diferente ``dato'' versus ``información'' porque será el mismo proceso de la investigación comercial el que permita esa transformación. Ahora que en la literatura de marketing estamos tan acostumbrados a oír el término valor, esta es la expresión del mismo en el proceso. Los profesionales de la investigación de mercados transforman el dato en información.

A partir de 2006, la AMA a través de su portal www.marketingpower.com define la investigación como\ldots{}

\begin{quote}
\emph{``la identificación, recopilación, análisis, difusión y uso sistemático y objetivo de la información, con el propósito de mejorar la toma de decisiones relacionadas con la identificación y solución de problemas y oportunidades de Marketing''}.
\end{quote}

Observamos algunos aspectos que son de gran relevancia. La palabra ``sistemático'' confiere una gran importancia al uso del método y a la planificación anticipada de todos los pasos a dar. Su uso del método científico se refleja en que se obtienen datos que nos sirven para testar hipótesis o planteamientos que previamente nos hemos establecido. La palabra ``objetivo'' confiere un aura de imparcialidad y de precisión en la toma del dato y debe servir para minimizar y/o eliminar cualquier tipo de sesgo que se pudiera producir en su materialización. Por último ya se atisban las diferentes etapas de la función de la investigación comercial que vemos en epígrafes posteriores.

\hypertarget{la-funciuxf3n-de-la-investigaciuxf3n-en-marketing}{%
\section{La función de la investigación en Marketing}\label{la-funciuxf3n-de-la-investigaciuxf3n-en-marketing}}

Del concepto de investigación de mercados, deducimos que la función de la investigación (comercial o de mercados) es proveer de información que asistirá a los managers de marketing de la empresa para que sean capaces de reconocer y reaccionar a las oportunidades de marketing y a los problemas que en el mercado se pudieran ocasionar (Hawkins y Tull, 1993). Dotar de un entorno de certidumbre en el cual se puedan tomar decisiones comerciales que afectaran a la estrategia de marketing tanto operativo como estratégico en la empresa.

Los puntos clave de esta funcionalidad son:

\begin{itemize}
\tightlist
\item
  Proveer información, aportar, sumar, permitir ver claramente a través de los ojos de los sujetos investigados.
\item
  Ayuda a los managers de marketing de la empresa a manejar, identificar signos, tendencias comportamientos, a explicarlos.
\item
  Aportar certidumbre, que nuestras hipótesis se vean refrendadas con datos del mercado
\item
  Anticipar el futuro, qué pasará si\ldots?
\end{itemize}

Cuando una empresa tiene una hipótesis y requiere disponer de datos sobre ella, es decir, testar en el mercado, tiene diferentes opciones. Por un lado puede analizar fuentes de información ya elaboradas en otros momentos del tiempo, con un objetivo diferente al que se le presenta ahora al investigador. Esas fuentes de información son las llamadas secundarias. Suelen estar disponibles porque son de carácter público o de posible compra: anuarios, informes de los centros de estudios económicos, estudios sectoriales, informes de investigaciones anteriores. Estas fuentes secundarias, pueden ser localizadas dentro de la misma empresa en la que trabajamos o vamos a trabajar (fuentes internas) o en lugares externos a la propia empresa para la que trabajamos o vamos a trabajar (fuentes externas).

Sin embargo, en ocasiones creemos que los datos que deseamos, no están a nuestro alcance o simplemente no existen. Esta es la situación en la cuál requerimos a las denominadas fuentes primarias, es decir, cuando decidimos que ante la ausencia de la misma vamos a fabricarla nosotros mismos con nuestra tarea de investigación: diseñemos nuestra propia metodología de investigación que irremediablemente acabará con la redacción de un informe que contendrá los datos que nosotros requerimos. También en este caso, aunque menos habitual se puede hacer esa distinción de fuentes primarias internas, cuando la información se generará en el seno de la empresa para la que trabajamos o externas (el caso más habitual) cuando salimos a su búsqueda o captación en el entorno de nuestra empresa.

\hypertarget{quuxe9-yo-cuxf3mo-debe-ser-la-informaciuxf3n-que-obtengamos}{%
\subsection{¿Qué y/o cómo debe ser la información que obtengamos?}\label{quuxe9-yo-cuxf3mo-debe-ser-la-informaciuxf3n-que-obtengamos}}

No sirve sin embargo cualquier tipo de dato para obtener información. El investigador, aplicará de forma sistemática y con una planificación muy metódica aquellas condiciones que permitan que el dato obtenido en la investigación sea útil y provechoso para los intereses de todos los participantes en la investigación. Del mismo modo, el factor tiempo es esencial en la investigación comercial. Un dato que tarda en recogerse puede ser un dato no válido en el momento de la interpretación y/o validación. Los datos deben ser actuales, fáciles de coleccionar (ágil su recogida) y posible su recogida de forma muy rápida, no permitiendo que la dilatación del período de investigación a lo largo del tiempo pueda introducir un factor de variabilidad del fenómeno estudiado.

\hypertarget{duxf3nde-obtenemos-los-datos}{%
\subsection{¿Dónde obtenemos los datos?}\label{duxf3nde-obtenemos-los-datos}}

Cualquier ámbito es bueno para la colección de datos en la empresa:

\begin{itemize}
\tightlist
\item
  datos de tipo comercial como cuotas de mercado, potencial de mercado, tendencias, evolución de las ventas \ldots{}
\item
  datos de cliente (perfiles, nichos) y/o competencia (quién, cómo, cuando, dónde\ldots)
\item
  datos de tipo financiero como márgenes comerciales, resultados netos, rentabilidad \ldots{}
\item
  datos del entorno político, tecnológico, social y/o legal como normas, tecnologías, factores del entorno.
\end{itemize}

A todos estos ámbitos podemos acudir con la investigación para tratar de alcanzar los objetivos que nos hayamos marcado. Sin embargo, no en todos los ámbitos y/o con los mismos objetivos, trataremos de alcanzarlos igual. El investigador dispone de distintos enfoques le permiten acercarse al problema / oportunidad con el objetivo de poder identificarlo/a y/o solucionarlo/aprovecharla

El enfoque denominado exploratorio permite comprender y conocer las bases de la naturaleza general de un problema o identificación de variables relevantes. Son significativos cuando el investigador no tiene suficiente comprensión de un fenómeno. Son trabajos de investigación que en mucho casos se consideran el primer paso o los preliminares de investigaciones en las que los datos iniciales son pocos. Sin embargo son flexibles y muy poco formalizados, lo que permite al investigador actuar con una sistemática ad-hoc al problema u oportunidad planteada. Sus requerimientos estadístico matemáticos con muy laxos. Se busca más la subjetividad y/o la interpretación de los hechos que la descripción precisa o cuantificable de una realidad. Este tipo de investigación se suele acometer con el uso de información secundaria y de técnicas cualitativas.

Alternativamente al anterior, nos encontramos con el enfoque descriptivo. Es un enfoque en el investigador, ya conocedor en profundidad del problema u oportunidad que se le presenta, trata de describir algo y por ende profundizar aún más en la naturaleza de un problema. Son métodos de investigación mucho más estructurados, formales y sistemáticos y habitualmente están basados en hipótesis o cuestiones a investigar. Se trazan una serie de objetivos precisos y cuantificables al alcanzar. Pueden ser transversales, produciendo en este caso un corte en el tiempo con una toma de datos o longitudinales, donde la toma de datos se sucede en olas programadas donde lo importante es conocer la evolución de la tendencia. Se caracterizan por un uso de técnicas cuantitativas.

Por último, los enfoques causales, donde el objetivo no esta tanto la descripción de una realidad, sino el estudio de las causas y los efectos que se producen con la experimentación de la introducción de estímulos en entornos naturales y/o artificiales. Se busca por tanto las relaciones causa-efecto entre variables y la catalogación de la función de cada una de ellas. Determina y es muy útil para conocer qué variables son causa (variables independientes) y cuáles resultado (variables dependientes). Normalmente es un enfoque basado en hipótesis planteadas a priori con un uso normal de técnicas cuantitativas y también un uso de la experimentación.

Estos tres enfoques están relacionados entre sí: es habitual ante el desconocimiento, un estudio exploratorio cuando se sabe poco del problema a investigar y entonces la exploración se convierte en el punto de partida del diseño de investigación que seguirá con estudios descriptivos o causales.

\hypertarget{tipologuxedas-en-la-investigaciuxf3n}{%
\section{Tipologías en la investigación}\label{tipologuxedas-en-la-investigaciuxf3n}}

Diferentes tipologías o taxonomías de la investigación de mercados se pueden hacer. Las iremos desarrollando a través de este documento según seamos capaces de entender los diferentes criterios en que se basan las mismas. Por ahora, las dejamos identificadas.

Para empezar, y como hemos realizado una muy básica introducción de qué es la investigación de mercados, nos interesa conocer su clasificación en virtud de que se desee identificar o solucionar un problema tenemos:

Si se busca la identificación del problema, las áreas o tipos de trabajo son\ldots{}

\begin{itemize}
\tightlist
\item
  potencial del mercado
\item
  participación en el mercado
\item
  imagen de marca
\item
  caracterización del mercado
\item
  análisis de ventas
\item
  tendencia del mercado
\item
  pronóstico del mercado
\end{itemize}

Si se busca la solución de problema \ldots{}

\begin{itemize}
\tightlist
\item
  segmentación
\item
  producto
\item
  asignación / sensibilidad de precios
\item
  promoción
\item
  comunicación
\item
  distribución
\end{itemize}

Por otro lado, si utilizamos un criterio de utilización de muestras y/o repetición de la investigación:

\begin{itemize}
\tightlist
\item
  Longitudinal, mediciones reiteradas sobre la misma o distinta muestra a lo largo del tiempo.
\item
  Transversal simple, realiza una medición sobre una muestra en un momento del tiempo.
\item
  Transversal múltiple, realiza una medición sobre diversas muestras.
\end{itemize}

En función o según el tipo de nivel de elaboración de la información:

\begin{itemize}
\tightlist
\item
  De campo, utiliza información sin elaborar que se debe conseguir.
\item
  De gabinete, utiliza información ya elaborada.
\end{itemize}

Por último, según el tipo de información utilizada:

\begin{itemize}
\tightlist
\item
  Cuantitativa, utiliza información de tipos numérica o cuantificable; el objetivo es cuantificar y extrapolar.
\item
  Cualitativa, utiliza información no fácilmente cuantificable; el objetivo no es cuantificar sino profundizar, comprender.
\end{itemize}

\hypertarget{la-industria-de-la-investigaciuxf3n}{%
\section{La industria de la investigación}\label{la-industria-de-la-investigaciuxf3n}}

Aunque en el tema 03, analizaremos más detenidamente muchos de los componentes de este epígrafe, valga esta primera aproximación para tener una idea general.

La industria de la investigación de mercados es una industria en la que participan diferentes actores que interactúan entre ellos, bajo una serie de normas de carácter superior. El centro de la industria es el cliente. El cliente (empresa privada o administración pública) de la investigación es quien habitualmente contrata la investigación a un instituto o agencia de investigación que son las empresas cuya actividad principal es la ejecución de la investigación. Sin embargo, hemos pasado de un final de siglo XX donde había una gran cantidad de empresas pequeñas de investigación, gabinetes de profesionales con una pequeña red de campo a, en este primera parte del XXI un mercado totalmente concentrado donde las 20 mayores empresas del mercado, copan más del 75 \% de la industria y donde las multinacionales de la investigación han tomado su puesto en la parrilla de salida. Todo esto se ha realizado por absorciones y fusiones, dado que es un mercado con grandes barreras de entrada en cuanto a tecnología y experiencia de mercado.

Sin embargo hay otro rasgo característico. De la tendencia de finales de los 90 donde las empresas del mercado abarcaban toda la funcionalidad de investigación, léase planteamiento, diseño, ejecución y difusión de la información, en este siglo XXI esa funcionalidad migra hacia la subcontratación, y así aparecen nuevos (en algunos casos los mismos) actores en la industria en diferentes papeles pero especialistas en su actividad: tecnológicas de software, centros de cálculo, redes de campo. Por último, no hay que olvidar a las grandes consultoras de marketing que en muchos casos también desarrollan tareas de investigación aunque no como actividad principal sino subcontratando a especialistas en investigación (institutos, agencias o funciones particulares) aquellas tareas que no pueden o saben asumir directamente.

Todo este arco de relaciones se establecen en un marco donde hay unas organizaciones y una normal que vela por la ética y buen hacer en la industria. Así encontramos a \href{https://www.aedemo.es}{AEDEMO (Asociación Española de Estudios de Mercado y Opinión)}. AEDEMO es la Asociación de los profesionales que desarrollan su actividad en la Investigación de Mercados, el Marketing y los Estudios de Opinión. El objetivo fundamental de AEDEMO es la difusión y control de las técnicas empleadas en la Investigación Comercial. Las actividades de AEDEMO se desarrollan en los campos de la formación, las publicaciones profesionales, los servicios a los asociados y las relaciones internacionales. Los objetivos de la Asociación son:

\begin{itemize}
\tightlist
\item
  Contribuir al conocimiento y desarrollo de las técnicas relativas a los Estudios de Mercado de Opinión, de Investigación Comercial y Marketing.
\item
  Promover los métodos necesarios para la correcta aplicación de los datos proporcionados por la Investigación de Mercados a los programas de comercialización de empresas.
\item
  Facilitar la información y mejora de los especialistas en Estudios de Mercados, Marketing y Opinión.
\item
  Colaborar en la solución de los profesionales de sus asociados.
\item
  Fomentar las relaciones de trabajo e interprofesionales de sus miembros.
\item
  Arbitrar las situaciones de conflicto técnico entre empresas y proveedores de servicios profesionales de Investigación de Mercados que sean aceptadas por ambas partes.
\end{itemize}

También encontramos \href{http://www.aneimo.com/}{ANEIMO (Asociación de Empresas de Investigación de Mercados y Opinión)} que es la asociación que aglutina a las empresas líderes del sector, representándolas en los diferentes ámbitos sociales y profesionales, promoviendo su desarrollo y asegurando que sus trabajos se realizan con altos estándares de calidad y siguiendo los códigos de ética profesional. Las empresas de ANEIMO cuentan con sistemas de calidad que ofrecen las máximas garantías sobre los trabajos que realizan. Según su actividad estarán certificadas por la norma UNE-ISO 20252 (Investigación de Mercado, Social y Opinión), la norma ISO 26362 (Access paneles in market, opinion and social research), comités de clientes o usuarios, otras normativas específicas y en algunos casos por la norma ISO 9001. Todas ellas están adheridas igualmente al Código Internacional CCI/ESOMAR para la Investigación Social y de Mercados.

En el año 2020, ambas organizaciones decidieron unir esfuerzo y se creó \href{http://ia-espana.es/}{Insights-Analytics España}, conocida como I+A, es la nueva asociación que integra Aneimo y Aedemo. La ambición de I+A es ampliar el enfoque tradicional de la investigación de mercados a las nuevas fuentes de datos, metodologías y disciplinas que han aparecido en los últimos años como consecuencia de la progresiva digitalización de la sociedad. La nueva asociación pretende dar voz todos sus socios, que serán empresas, profesionales, clientes y académicos, para seguir impulsando el importante rol que la investigación de mercados tiene en el conocimiento de los consumidores y la sociedad. Hasta la integración jurídica definitiva de Insights + Analytics España, los puntos de información seguirán siendo los de ambas asociaciones fundadoras.

Por último, la \href{https://www.aepd.es/es}{APD (Agencia de Protección de Datos)} que es una entidad pública, creada en virtud del Reglamento de la LOPD (Ley Orgánica 15/1999, de 13 de diciembre, de Protección de Datos de Carácter Personal) que junto con la LSSI (Ley 34/2002 de 11 de julio de Servicios de la Sociedad de la Información y del Comercio Electrónico) conforman el marco normativo legal de la investigación de mercados cumplimentado con el Código Internacional CCI de ESOMAR.

\hypertarget{tema02}{%
\chapter{Diseño de la investigación}\label{tema02}}

Como hemos venido indicando en los capítulos anteriores de introducción, una investigación requiere de un buen diseño metodológico y esto es algo que sólo se puede producir tras el briefing con el cliente y cuando se tienen claros los objetivos a alcanzar. Denominamos diseño de investigación al esquema o programa donde específicamente se detallan los pasos a seguir para obtener la información requerida para estructurar y/o resolver los problemas / oportunidades de investigación de mercados.

Un diseño óptimo y correcto, garantiza la realización eficaz y eficiente del proyecto de investigación de mercados. Este esquema se puede estructura en diferentes etapas que confluyen al final en la presentación de la oferta de investigación a nuestro cliente:

\begin{enumerate}
\def\labelenumi{\arabic{enumi}.}
\tightlist
\item
  Definir el problema de investigación
\item
  Estimar el valor de la información que será provista por la investigación
\item
  Seleccionar el modo de obtención de la información
\item
  Seleccionar el método de medida
\item
  Seleccionar el proceso de muestreo
\item
  Seleccionar la propuesta de análisis
\item
  Evaluar los principios éticos de la investigación
\item
  Especificar el tiempo y coste financiero de la investigación
\item
  Preparar la propuesta de investigación
\end{enumerate}

Al definir el problema de investigación nos referimos al especificar qué información será requerida para alcanzar los objetivos propuestos en el briefing con el cliente. Este paso determinará a qué tipo de enfoque de investigación nos dirigimos: exploratorio, descriptivo o causal. El segundo paso, nos permite utilizar de forma juiciosa un criterio de evaluación coste/beneficio o valor esperado de la información a obtener estimado el mismo en función de los diferentes niveles de precisión de la misma.

Determinados los objetivos y sus necesidades de información y decidido cuál va ser el enfoque y las técnicas que damos a nuestra investigación, el tercer paso se refiere al la decisión de como vamos a recoger esa información. Debemos determinar si datos secundarios, un sondeo o un experimento producirán los datos requeridos. El cuarto punto nos debe dirigir a elegir como usaremos los cuestionarios, técnicas proyectivas, observación o cualquier otra técnica de investigación para medir nuestros datos.

En el quinto punto determinaremos quien y cuántos elementos responderán a nuestra demanda de información, mientras que el sexto paso determina los procesos estadísticos apropiados para analizar los datos que requerimos; en el séptimo paso, revisamos todos los aspectos de la investigación para que sean todos acorde a la ética que nos impone nuestro código deontológico. El el octavo paso y para cada elemento de investigación, desarrollamos los tiempos y los costes financieros estimados y los comparamos con el valor estimado de la información y las restricciones de tiempo impuestas por el problema a analizar.

Por último, en el noveno paso desarrollamos la llamada propuesta de investigación.

Tres son los enfoques típicos de investigación que dan lugar al desarrollo de los diferentes tipos de investigación: cualitativa o cuantitativa:

\begin{itemize}
\tightlist
\item
  Investigación exploratoria
\item
  Investigación descriptiva
\item
  Investigación experimental
\end{itemize}

\hypertarget{investigaciuxf3n-exploratoria}{%
\subsection{Investigación exploratoria}\label{investigaciuxf3n-exploratoria}}

Definimos un diseño de investigación exploratoria a aquel en el que lo importante es explorar o examinar un problema o situación para brindar conocimiento y comprensión. Se caracteriza porque la información necesaria se define vagamente, dado que el investigador no tiene mucho conocimiento sobre aquello que están investigando. Su objetivo es buscar respuestas. Así el proceso de investigación es flexible y no estructurado, se va adaptando sobre la marcha a las necesidades del investigador. Este tipo de investigación que se ajusta mucho a la caracterización del enfoque exploratorio, utiliza muestras muy reducidas y se basa en el análisis cualitativo, por tanto mucho más subjetivo de los datos.
Los métodos de investigación más habituales en este enfoque son las entrevistas con expertos, las encuestas piloto, las sesiones de grupo, las técnicas proyectivas y/o la observación cualitativa. Todas estas técnicas, arrojan unos resultados que son tentativos, como un examen previo para tantear la cuestión, su profundidad y su alcance. En muchas ocasiones, este tipo de enfoque se usa para disponer de la información suficiente para poder continuar con un enfoque de tipo concluyente, bien descriptivo o causal. Permiten definir o formularios acciones de investigación más concretas o alternativas a las acciones inicialmente propuestas, de alguna forma consigue enfocar u orientar mejor el problema ayudando incluso en el desarrollo de hipótesis a tetar en la investigación
Un ejemplo de este tipo de enfoque podría ser la identificación de las causas sociales que según las personas debería preocupar a las empresas, dentro de un proyecto de potenciación de la RSC en una multinacional.

\hypertarget{investigaciuxf3n-descriptiva}{%
\subsection{Investigación descriptiva}\label{investigaciuxf3n-descriptiva}}

Del mismo modo, definimos el diseño descriptivo como aquel diseño de investigación en el que lo importante es establecer pruebas de hipótesis específicas y medición / cuantificación de elementos, su objetivo específico es describir características o funciones del mercado con precisión. Se caracteriza por definir previamente los objetivos y por tener un diseño poco flexible. La información necesaria es definida con muchas claridad y extrema precisión a partir de procesos de investigación formal sistemática y muy estructurado.

En este tipo de diseños se utilizan muestras grandes y representativas de la población de la cuál se extraen, utilizando el análisis de datos cuantitativo. Los resultados son concluyentes y objetivos, lo que facilita la toma de decisiones dado que hay una descripción precisa de las magnitudes que son objetivo de estudio.

Ejemplos de este tipo de enfoque es el uso de datos secundarios analizados de forma cuantitativa, las encuestas ad-hoc, los observatorios, trackings, barómetros, la observación cuantitativa o también estudios de tipo panel con análisis de cuotas de venta, notoriedad de marcas o sensibilidad de precios.

Los enfoques en investigación o con enfoque descriptivo son a los más divulgados en la disciplina del marketing. Así nos encontramos con diseños obtienen de una sola vez la información de una muestra dada de elementos de la población o también denominados diseños transversales. Son lo más utilizados, pero no los que tienen un mayor volumen de negocio.

Aquí englobamos tanto los diseños transversales simples, donde se extrae una sola muestra de encuestados en la población objetivo y se extrae información de la misma en una única ocasión; haciendo un símil se puede hablar de estudios que realizan ``una foto de la situación'' y también tenemos diseños transversales múltiples, donde hay dos o más muestras de encuestados y de cada muestra se obtiene información de una sola vez. Hay cierta bibliografía que los cataloga como longitudinales, dado que se alargan en el tiempo. Por último, los longitudinales, donde se produce una medición repetida sobre la misma muestra a lo largo del tiempo en períodos de la misma duración.

\hypertarget{investigaciuxf3n-experimental}{%
\subsection{Investigación experimental}\label{investigaciuxf3n-experimental}}

Por último los enfoques experimentales son aquellos cuyo objetivo determinar relaciones causa-efecto obteniendo evidencias de las mismas. Se caracterizan porque utilizan la manipulación de variables independientes, siendo con controlados los efectos de esa manipulación por medio de las llamadas variables dependientes o de control con un carácter mediador. Se persigue determinar las relaciones de dependencia entre estas variables, cuáles de ellas son causa (independientes) y cuáles son efecto (dependientes).

Los métodos de recolección utilizados están basados en las mismas técnicas que los enfoques o diseños descriptivos.

\hypertarget{relaciuxf3n-entre-enfoques}{%
\subsection{Relación entre enfoques}\label{relaciuxf3n-entre-enfoques}}

Atendiendo a Malhotra (2008) los enfoques se relacionan entre sí. Cuando se sabe poco de aquel problema que se va a analizar, se utiliza la investigación exploratoria. Pero en muchas ocasiones, ésta suele continuarse con la investigación descriptiva o causal, una vez se haya adquirido el conocimiento necesario para precisar mejor el problema a analizar.

La decisión de comenzar con un enfoque de tipo exploratorio depende del grado de precisión con que se haya definido el problema inicial. Como ya hemos mencionado, si el problema está poco documentado o es poca la información, será inevitable es inicio.

Pero también se puede dar la situación contraria, aquella en la que a partir de ciertos resultados que se obtienen de la realidad de una descripción o experimentación, se desea ampliar conocimiento o se intuye algo que no se comprende.

En estos casos, se busca profundizar en el problema y por tanto, a ese enfoque descriptivo o ese enfoque causal le seguiría una aplicación de un enfoque exploratorio. Esta situación, sin embargo es mucho más atípica que la anterior.

\hypertarget{fuentes-de-informaciuxf3n-interna-y-externa}{%
\section{Fuentes de información interna y externa}\label{fuentes-de-informaciuxf3n-interna-y-externa}}

Ya hemos visto que el tercer paso en un buen diseño de investigación es el planteamiento de que selección de fuentes de información realizamos, y por tanto, es necesario acudir a fuentes de información para recopilar todo aquello que nos pudiera hacer falta para tener los datos necesarios con los que plantear una metodología correcta, tras la fijación de los objetivos de la investigación realizada en el briefing con el cliente.

Así, nos enfrentamos a las que denominamos fuentes secundarias o datos ya existentes y generados con otra finalidad a la nuestra, bien sea específica o bien general, diferente al problema o necesidad de información que el investigador pretende abordar pero que le ayudan a enmarcar la investigación en un ámbito de conocimiento.

Nos referimos a libros y publicaciones periódicas libros, revistas profesionales, publicaciones sectoriales o a informes, fuentes estadísticas y bases de datos de organismos públicos o privados Empresas e institutos de investigación de mercados. Estas fuentes se pueden localizar en:

\begin{itemize}
\tightlist
\item
  MINISTERIOS
\item
  ORGANISMOS DE LAS CC.AA.
\item
  INSTITUTO NACIONAL DE ESTADÍSTICA
\item
  INSTITUTO DE COMERCIO EXTERIOR
\item
  INSTITUTO NACIONAL DE CONSUMO
\item
  CÁMARAS DE COMERCIO
\item
  CENTROS DE DOCUMENTACIÓN PÚBLICOS Y/O PRIVADOS
\item
  SERVICIOS DE ESTUDIOS DE LAS ENTIDADES FINANCIERAS
\item
  \ldots{}
\end{itemize}

No obstante, dado que en los tiempos actuales el acceso a la información se ha generalizado y facilitado tanto, es muy importante que ante el aluvión de información y datos con que nos encontraremos ante la resolución de un problema, hay que plantearse realizar una evaluación de os datos y/o informes adquiridos:

\begin{itemize}
\tightlist
\item
  ¿Los datos ayudan a contestar las cuestiones establecidas en la definición del problema?
\item
  ¿Los datos son aplicables al período de interés?
\item
  ¿Los datos son aplicables a la población de interés?
\item
  ¿Otras clasificaciones de términos y variables son aplicables al proyecto actual?
\item
  ¿Las unidades de medida son comparables?
\item
  ¿Es la fuente original de los datos?
\item
  El coste de adquisición de datos, ¿merece la pena?
\end{itemize}

\hypertarget{la-propuesta-de-investigaciuxf3n}{%
\section{La propuesta de investigación}\label{la-propuesta-de-investigaciuxf3n}}

No debemos perder de vista que el briefing nos debe conducir a la presentación de una oferta. El investigador deberá acabar por comunicar al cliente qué y cómo va abordar los objetivos que se han definido. En la práctica, las consideraciones vistas con anterioridad se interrelacionan unas con otras, y no se suceden en una secuencia ordenada; no es necesario que la composición de la lista de puntos a revisar siga el mismo orden en que éstos han surgido y se han acordado. Su función es facilitar la continua revisión \emph{desde el principio} para asegurarse que todos los puntos importantes se han cubierto antes del acuerdo final de la investigación. El uso efectivo de la lista debe depender de la clara definición de los objetivos del proyecto y de una completa evaluación de los datos preliminares.

No todos los elementos expuestos son relevantes para todos los tipos de proyectos de investigación; la expresión ``cuando sea relevante'' debe aplicarse a lo largo de toda la lista de puntos a revisar y deben ser mencionados en el documento de oferta.

No seremos capaces de completar una propuesta de investigación si no conocemos el abanico de posibilidades a nuestro alcance, por lo que será necesaria una lectura de todos los temas restantes para poder llegar a desarrollar el documento de oferta o propuesta de investigación.

Aquí dejamos los apartados que conforman lo que sería el guión general de una propuesta de investigación, siguiendo las recomendaciones de la ISO 20252.

\begin{enumerate}
\def\labelenumi{\arabic{enumi}.}
\tightlist
\item
  Resumen ejecutivo
\item
  Antecedentes de la investigación
\item
  Definición del problema de investigación
\item
  Enfoque seleccionado
\item
  Diseño metodológico de la investigación e instrumentos de medición propuestos acuerdo a objetivos de 1. investigación
\item
  Descripción de la propuesta de trabajo de campo y recolección de datos
\item
  Descripción de la propuesta de análisis de datos
\item
  Propuesta de redacción y entrega de informe
\item
  Calendario y Presupuesto
\item
  Apéndices
\end{enumerate}

***********Enlace a las propuestas de todos los años.*********

\hypertarget{tema03}{%
\chapter{Técnicas de Investigación}\label{tema03}}

We describe our methods in this chapter.

\hypertarget{tema04}{%
\chapter{Medición y escalas}\label{tema04}}

Some \emph{significant} applications are demonstrated in this chapter.

\hypertarget{example-one}{%
\section{Example one}\label{example-one}}

\hypertarget{example-two}{%
\section{Example two}\label{example-two}}

\hypertarget{tema05}{%
\chapter{Diseño de cuestionarios}\label{tema05}}

We have finished a nice book.

\hypertarget{tema06}{%
\chapter{Diseño y proceso muestral}\label{tema06}}

We have finished a nice book.

\hypertarget{tema07}{%
\chapter{Trabajo de campo}\label{tema07}}

We have finished a nice book.

\hypertarget{tema08}{%
\chapter{Proceso de datos}\label{tema08}}

We have finished a nice book.

  \bibliography{book.bib,packages.bib}

\end{document}
